\documentclass[11pt]{article}
\usepackage{html}

\title{Using the Jimple Parser}
\author{Patrice Pominville\\ \htmladdnormallink{patrice@sable.mcgill.ca}{mailto:patrice@sable.mcgill.ca}}
\date{March 24, 2000}

\begin{document}
\maketitle

\section{Introduction}
At the heart of the Soot bytecode optimization framework is the Jimple
Internal Representation.  This document describes Soot's ability to
read in textual Jimple files.

We have already seen Soot's ability to load Java {\tt .class} files.
Soot can output the Jimple representation of classes in a textual
format ({\tt .jimple} files) and reread this textual format back into
Soot.  

The ability to parsing textual {\tt .jimple} files is quite handy. For
example, a user could output classfiles as {\tt .jimple} files, tweak
these by hand, and then use Soot to parse them and output tweaked
bytecode.  Soot thus provides a high-level facility for editing class
files.  Another potential application of the {\tt .jimple} files is to
cache optimized classfiles for future analysis. Soot can, at a
later time, read these files back in, and will not need to reoptimize
the original classfile.  This leads to faster compilation times.
Finally, a user, or a software tool, can write a class in the Jimple
textual format, use Soot to parse the Jimple, and thus create bytecode. 

The Jimple format is self-explanatory. The SableCC grammar for textual
Jimple can be found in the source code distribution of Soot; it is called
{\tt jimple.scc}. The following is an fragment of Jimple code:

\begin{verbatim}
    {
        java.lang.String[] r0;
        long l0, $l1, $l2, $l3, $l4, $l5, $l6, $l7;
        int i0, $i1, $i2, i3;
        java.io.PrintStream $r1, $r2, $r3;

        r0 := @parameter0: java.lang.String[];
        i0 = 0;
        $i1 = 33 % 33;
        $i2 = $i1 + 2;
        if $i2 != 43 goto label0;

        i3 = 9;
        goto label1;

     label0:
        i3 = 4;
\end{verbatim}

\section{Producing {\tt .jimple} Files}

The default output format for Soot is classfiles. However, Soot can
also produce output in other formats, i.e. textual Jimple. To output
{\tt .jimple} files, invoke Soot using the {\tt -f jimple} command line
option. For example,

\begin{verbatim}
  java soot.Main -f jimple --app test
\end{verbatim}

will output test and dependent classes as {\tt .jimple} files.

\section{Parsing {\tt .jimple} Files}

Out of the box, Soot will first look for a classfile to find the
definition of a given class; only if this fails will it look for a
{\tt .jimple} definition. The command line option {\tt -}{\tt -src-prec
jimple} (source precedence) instructs Soot to preferentially resolve
the class from a {\tt .jimple} file, falling back on {\tt .class}
files.  When resolving, Soot will look for {\tt .jimple} files using
the {\tt soot-classpath} in the same manner as it does to find {\tt
.class} files. For example, to build on the previous example, the
command

\begin{verbatim}
  java soot.Main --src-prec jimple --app test
\end{verbatim}

will read the class {\tt test} from the file {\tt test.jimple}, and its
related classes from the other {\tt .jimple} files produced in the previous
step.

For more details on the {\tt -f} and {\tt -}{\tt -src-prec} options
the user can refer to the Soot \htmladdnormallink{command line documentation}{../usage}.

\section*{History}
\begin{itemize}
\item March 24, 2000: Initial version.
\item May 31, 2003: Updated for Soot 2.0.
\end{itemize}

\end{document}

